% A LaTeX template for EXECUTIVE SUMMARY of the MSc Thesis submissions to
% Politecnico di Milano (PoliMi) - School of Industrial and Information Engineering
%
% P. F. Antonietti, S. Bonetti, A. Gruttadauria, G. Mescolini, A. Zingaro
% e-mail: template-tesi-ingind@polimi.it
%
% Last Revision: October 2021
%
% Copyright 2021 Politecnico di Milano, Italy. Inc. All rights reserved.

\documentclass[11pt,a4paper]{article}

%------------------------------------------------------------------------------
%	REQUIRED PACKAGES AND  CONFIGURATIONS
%------------------------------------------------------------------------------
% PACKAGES FOR TITLES
\usepackage{titlesec}
\usepackage{color}

% PACKAGES FOR LANGUAGE AND FONT
\usepackage[utf8]{inputenc}
\usepackage[english]{babel}
\usepackage[T1]{fontenc} % Font encoding

% PACKAGES FOR IMAGES
\usepackage{graphicx}
\graphicspath{{Images/}} % Path for images' folder
\usepackage{eso-pic} % For the background picture on the title page
\usepackage{subfig} % Numbered and caption subfigures using \subfloat
\usepackage{caption} % Coloured captions
\usepackage{transparent}

% STANDARD MATH PACKAGES
\usepackage{amsmath}
\usepackage{amsthm}
\usepackage{bm}
\usepackage[overload]{empheq}  % For braced-style systems of equations

% PACKAGES FOR TABLES
\usepackage{tabularx}
\usepackage{longtable} % tables that can span several pages
\usepackage{colortbl}

% PACKAGES FOR ALGORITHMS (PSEUDO-CODE)
\usepackage{algorithm}
\usepackage{algorithmic}

% PACKAGES FOR REFERENCES & BIBLIOGRAPHY
\usepackage[colorlinks=true,linkcolor=black,anchorcolor=black,citecolor=black,filecolor=black,menucolor=black,runcolor=black,urlcolor=black]{hyperref} % Adds clickable links at references
\usepackage{cleveref}
\usepackage[square, numbers, sort&compress]{natbib} % Square brackets, citing references with numbers, citations sorted by appearance in the text and compressed
\bibliographystyle{plain} % You may use a different style adapted to your field

% PACKAGES FOR THE APPENDIX
\usepackage{appendix}

% PACKAGES FOR ITEMIZE & ENUMERATES
\usepackage{enumitem}

% OTHER PACKAGES
\usepackage{amsthm,thmtools,xcolor} % Coloured "Theorem"
\usepackage{comment} % Comment part of code
\usepackage{fancyhdr} % Fancy headers and footers
\usepackage{lipsum} % Insert dummy text
\usepackage{tcolorbox} % Create coloured boxes (e.g. the one for the key-words)
\usepackage{stfloats} % Correct position of the tables
\usepackage{multirow}
\usepackage{multicol}



%-------------------------------------------------------------------------
%	NEW COMMANDS DEFINED
%-------------------------------------------------------------------------
% EXAMPLES OF NEW COMMANDS -> here you see how to define new commands
\newcommand{\bea}{\begin{eqnarray}} % Shortcut for equation arrays
\newcommand{\eea}{\end{eqnarray}}
\newcommand{\e}[1]{\times 10^{#1}}  % Powers of 10 notation
\newcommand{\mathbbm}[1]{\text{\usefont{U}{bbm}{m}{n}#1}} % From mathbbm.sty
\newcommand{\pdev}[2]{\frac{\partial#1}{\partial#2}}
% NB: you can also override some existing commands with the keyword \renewcommand

%----------------------------------------------------------------------------
%	ADD YOUR PACKAGES (be careful of package interaction)
%----------------------------------------------------------------------------
\usepackage{amsfonts} 

%----------------------------------------------------------------------------
%	ADD YOUR DEFINITIONS AND COMMANDS (be careful of existing commands)
%----------------------------------------------------------------------------


% Do not change Configuration_files/config.tex file unless you really know what you are doing.
% This file ends the configuration procedures (e.g. customizing commands, definition of new commands)
\input{Configuration_files/config}

% Insert here the info that will be displayed into your Title page
% -> title of your work
\renewcommand{\title}{Title}

% -> author name and surname
\renewcommand{\author}{Andrea Bonifacio, Sara Gazzoni}
% -> MSc course
\newcommand\norm[1]{\lVert#1\rVert}
\newcommand{\course}{Advanced Programming for Scientific Computing}
% -> advisor name and surname
\newcommand{\advisor}{Stefano Pagani}
% IF AND ONLY IF you need to modify the co-supervisors you also have to modify the file Configuration_files/title_page.tex (ONLY where it is marked)
\newcommand{\firstcoadvisor}{Mattia Corti} % insert if any otherwise comment
%\newcommand{\secondcoadvisor}{Name Surname} % insert if any otherwise comment
% -> academic year
\newcommand{\YEAR}{2022-2023}

%-------------------------------------------------------------------------
%	BEGIN OF YOUR DOCUMENT
%-------------------------------------------------------------------------
\begin{document}

%-----------------------------------------------------------------------------
% TITLE PAGE
%-----------------------------------------------------------------------------
% Do not change Configuration_files/TitlePage.tex (Modify it IF AND ONLY IF you need to add or delete the Co-advisors)
% This file creates the Title Page of the document
\input{Configuration_files/title_page}

%%%%%%%%%%%%%%%%%%%%%%%%%%%%%%
%%     THESIS MAIN TEXT     %%
%%%%%%%%%%%%%%%%%%%%%%%%%%%%%%

%-----------------------------------------------------------------------------
% INTRODUCTION
%-----------------------------------------------------------------------------

\section{Introduction}
Full 3D blood flow models are important in the study of cardiovascular system 
since they allow to extract detailed quantities of interest but their actual 
implementation is limited due to their high computational cost. 
For this reason, reduced order models are widely used in this fields because of 
their efficiency. An example is presented in \cite{Luca}, where a one-dimensional
redueced order model is implemented to simulate the blood flow in the aorta using 
a graph neural network trained on three-dimensional simulations. In this work,
we propose a different application, where the graph neural network is used to 
approximate the solution of different problems. In particular, we consider the
heat equation as test case, but the goal of the project is to show the potential
extension of this approach to solve more difficult problems with complex geometries,
such as the simulations of proteins spreading in the neural system, which are at
the basis of neurodegenerative diseases \cite{MattiaCorti}.
The main part of this project is the implementation of a library for data generation 
used to train the graph neural network and the adaptation of the code [di Luca 
non so come citarlo] to make it suitable for our specific test case. In the following sections, 
we first present the problem formulation and a detailed description of the code 
developed, then we show the results obtained and a discussion of the possible 
further developments and extensions.

%-----------------------------------------------------------------------------
% Problem
%-----------------------------------------------------------------------------

\section{Problem overview}

% descrizione del problema generico Lu=F risolto con Fenics
We consider a general time-dependent variational problem of the form:
\[Lu=f\]
with \(L\) a linear operator, \(f\) a source term and \(u\) the solution. Given a specific 
geometry \(\Omega\) and using the finite element method implemented in Fenics, we can solve 
this problem and obtain the solution \(u^{n}\) at each time step \(n\). From this,
we can generate a graph that decribes the geometry of the problem and the solution, 
storing some values of interest as features of the nodes and the edges.
Solving the problem for different geometries and different values of the parameters
(e.g. the diffusivity constant) we can generate a dataset that will be used to train
the graph neural network. 
As in \cite{Luca}, the GNN is applied iteratively: at each time step it takes 
as input the system state \(\Theta^{n}\), which is the set of all the nodes and edges features 
at that time step, and it predicts an update for the state variables. The prediction is 
combined with the previous time step to estimate \(\Theta^{n+1}\).


% test case heat equation
\subsection{Test case: heat equation}
In this work, we consider the heat equation as test case. 
The mathematical formulation of the problem is the following:
\begin{equation}
    \begin{cases}
        \frac{\partial u }{\partial t} = k \Delta u \quad \text{in} \ \Omega \subset \mathbb{R}^2, \\
        \frac{\partial u}{\partial n} = h \quad \text{on} \ \partial \Omega_{inlet}, \\
        \frac{\partial u}{\partial n} = 0 \quad \text{on} \ \partial \Omega_{outlet} 
        \cup \partial \Omega_{walls}.
    \end{cases}
\end{equation}
where \(u\) is the temperature, \(k\) is the diffusivity constant, \(h\) is the
Neumann condition at the inlet boundary. As domain \(\Omega\) we consider different 
geometries such as the one shown in Figure \ref{geometry}: 
% spiegare meglio come sono fatte le mesh

We generated 20 different mesh using gmsh. 
Then we solved the problem in Fenics using Discontinuous Galerkin method and implicit
Euler for time discretization, imposing as Neumann condition at inlet 
\(h = 2e^{-(t-2.5)^2}\). 

% creazione dei grafi con le varie features 
% training della GNN con i parametri

%-----------------------------------------------------------------------------
% Code
%-----------------------------------------------------------------------------

\section{Code}


%-----------------------------------------------------------------------------
% Results
%--- --------------------------------------------------------------------------

\section{Results}

%-----------------------------------------------------------------------------
% Further work
%-----------------------------------------------------------------------------

\section{Further work}

%---------------------------------------------------------------------------
%  BIBLIOGRAPHY
%---------------------------------------------------------------------------
\newpage
% Remember to insert here only the essential bibliography of your work
\bibliography{bibliography.bib} % automatically inserted and ordered with this command

\end{document}
